\documentclass[letterpaper, 11pt]{article}
\usepackage[utf8]{inputenc}
\usepackage[margin=1.0in]{geometry}
\usepackage[final]{graphicx}
\usepackage{amsmath}
\usepackage{tgpagella}
\usepackage{mathpazo} 

\renewcommand\tablename{Tabla} 
\renewcommand\figurename{Figura} 


\begin{document}
\begin{center}
\noindent\makebox[\linewidth]{\rule{\textwidth}{0.4pt}}
\vspace{0.2cm}
\textsc{Electrónica Industrial - ELO 381}\\
\vspace{0.2cm}
\huge{\textsc{Título Informe}}\\
\vspace{0.6cm}
\normalsize{\textsc{Nombres: Nombre1, Nombre2, Nombre3 }}\\
\vspace{-0.2cm}
\noindent\makebox[\linewidth]{\rule{\textwidth}{0.4pt}}
\end{center}



\section*{Ejercicio}
En la Fig. \ref{fig:fig01} se muestra un rectificador puente completo con carga resistiva. Considere voltaje de entrada sinusoidal de amplitud 100V y una resistencia de salida de 10$\Omega$.
\begin{enumerate}\setlength{\itemsep}{3pt}
\item Calcule los valores de voltaje de salida, corriente de entrada rms y factor de potencia considerando que los diodos tienen una caída de tensión de 1V.
\item Calcule la eficiencia del rectificador. 
\item Compare los resultados con las ecuaciones del apunte obtenidas con diodos ideales.
\end{enumerate}

\begin{figure}[!ht]
\centering
\includegraphics[scale=1]{fig01.pdf}
\caption{Rectificador de diodos.}
\label{fig:fig01}
\end{figure}


\end{document}
